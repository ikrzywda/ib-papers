\documentclass{article}
\usepackage[utf8]{inputenc}
\usepackage{minted}
\usepackage{float}
\usepackage{hyperref}
\usepackage[margin=0.75in]{geometry}

\title{Criterion C}
\author{Word count: 974}
\date{}

\begin{document}

\maketitle
\tableofcontents

\section{Lexer}

    Lexer is the part of interpreter that converts text to tokens. Token is 
    a structure holding a string and identifier that is later used in parsing.
    General idea behind lexer has been shown in criterion B.
    Below is a concrete example of lexer working from source
    code showing lexical analysis for variables, mathod names and reserved
    keywords.
    \begin{minted}[linenos]{c++}
tk::Token &Lexer::id()
{
    int id = tk::ID_VAR;
    attr_buffer.push_back(c);

    if(!is_upcase(c)) id = tk::ID_METHOD;

    advance();

    while(std::isalnum(c) || c == '_')
    {
        if(!is_upcase(c)) id = tk::ID_METHOD;

        attr_buffer.push_back(c);
        advance();
    }
    if(tk::lookup_keyword(attr_buffer) > 0)
        id = tk::lookup_keyword(attr_buffer);

    token.mutate(id, attr_buffer, line_num);
    return token;
}   
    \end{minted}

    The starting point for this lexical analysis of id's are regular expressions
    for them.
    \begin{verbatim}
ID : [A-Z]([A-Z] | [0-9] | _)*
METHOD_ID : ([A-Z] | [a-z]) ([A-Z] | [a-z] | [0-9] _)*
    \end{verbatim}
    The function holds a state (\texttt{id}), which is the identifier
    for the token returned by the method. The starting state is variable id, because
    it is a subset of method id, as it is easier to change from subset to superset
    than the other way around. The loop iterates over the string buffer until it 
    detects character that is not defined for method id. After the loop, the token
    buffer is checked whether it is a reserved keyword. Method \texttt{lookup\_keyword}
    searches a map, whose key is a string, holding identification for reserved keyword. 
    Then token is changed to contain appropriate data and its address in stack
    is returned.

\section{Recursive-descend parser}
    
    The role of parser is to make a representiation of source code that can later
    be processed by the intepreter. In this case parser generates abstract
    syntax tree (AST) from tokens (terminals). Example of AST has been shown in 
    criterion B.
    The blueprint for parser is grammar, in this example, grammar for expressions.
    \begin{verbatim}
expr    :: term + term
         | term - term
         | term
    \end{verbatim}
    In recursive-descend parser, each nonterminal has its own method, which is
    responsible for building a subtree expressing a bit of syntax in the main AST.
    Below, is such method for above-mentioned expression.
    \begin{minted}[linenos]{c++}
ast::AST *Parser::expr()
{
    ast::AST *root, *new_node;
    root = term();

    while(token.id == tk::PLUS 
            || token.id == tk::MINUS)
    {
        new_node = new ast::AST(token, ast::BINOP);
        new_node->push_child(root);
        root = new_node;
        eat(token.id);
        new_node->push_child(term());
    }

    return root;
}
    \end{minted}
    We start by declaring two variables that will hold poitners to AST nodes.
    We assign \emph{root} pointer to \emph{term}, as it is the first nonterminal 
    that has to be evaluated in each of cases in the grammar. Then, if the next 
    token is neither \emph{PLUS}, nor \emph{MINUS}, we return the node. In other 
    case we enter the while loop that will turn as long as the token during the 
    check is \emph{PLUS} or \emph{MINUS}. Inside the loop, we assign a pointer to 
    a subroot to variable \emph{new\_node}, which is the terminal of binary 
    operation holding the operator. The node currently held by \emph{root} is 
    the left operand of binary operation and is linked to the root by funtion 
    \emph{push\_child} that appends the pointer to the vector inside the node 
    holding the children. After \emph{root} reclaiming the root of the tree, we 
    advance to the next token using method \emph{eat}, which checks whether the 
    current token is the same as requested token by a method. Because we already 
    checked if the token is correct, we just pass the id of current token. Lastly, 
    we append another \emph{term} to our root and either repeat the process or 
    break out of the loop and return the poitner to the tree. This method, along with 
    helper functions \emph{term} and \emph{factor} provided an example expression 
    $10 + 100 * 10 - 6$ will output following tree (actual output of the 
    interpreter with flag \emph{-p}):
    \begin{verbatim}
|--[-]               
    |--[+]        
        |--[10]   
        |--[*]     
            |--[100]
            |--[10] 
    |--[6]        
    \end{verbatim}


\section{Tree-walk interpreter}

    Tree-walk interpreter traverses the AST while executing operations that 
    are described by the terminals within the tree. Additionally it evaluates
    the semantics of the code and throws run-time errors.

\subsection{Evaluating and executing binary operations}

    Below is code evaluating and executing binary operations.

    \begin{minted}[linenos]{c++}
rf::Reference *Interpreter::binop
        (rf::Reference *l, rf::Reference *r, int op)
{
    int type = check_types(l, r);
    rf::Reference *out;
        if(type == tk::STRING)
        {
            if(op == tk::PLUS)
            { 
                out = add(l, r);
            }
            else
            { 
                delete l;
                error("cannot make this type of comparison on strings", r);
            }
    }
    else if(op == tk::MINUS || op == tk::MULT || op == tk::PLUS)
    {
        switch(op)
        {
            case tk::PLUS: { out = add(l, r); break; }
            case tk::MINUS:
                 {
                    out = new rf::Reference(l->token.val_num - r->token.val_num); 
                    break;
                 }
            case tk::MULT: 
                 {
                    out = new rf::Reference(l->token.val_num * r->token.val_num); 
                    break;
                 }
        }
    }
    else
    {
        out = divide(l, r, op);
    }
    delete l; delete r;
    return out;
}
    \end{minted}
    In order to evaluate a binary operation, \emph{binop} takes as arguments 
    left and right node of an expression subtree along with the operator. This 
    is because binary operation is bound to the right side of assignment, and 
    because there are more right-bound expressions, there is a function that 
    manages that choice so that functions working with terminals can be cleaner 
    and more readable. The first function, which is called is \emph{check\_types}, 
    returns the type of expression, but also has optional side effect of throwing 
    a run-time error when incompatible types are to be computed. When the type 
    check is successful, there are two main branches. If the type is a string, 
    the interpreter is only to execute concatenation, which means that only 
    viable operator is addition, which is tested in 6th line. If the types are 
    numerical, then all binary operations can be performed. Addition, subtraction 
    and multiplication are seperated from divisions and modulo because they do not 
    need checking for zero in right value. If any of the options checks out, a 
    new reference (a field in activation record) is created with literal expression 
    in its constructor. In the end, all redundant heap-allocated objects are 
    disposed of to prevent memory leaks and pointer to reference is returned.

\subsection{Executing function calls}

    The last example are function calls.

    \begin{minted}[linenos]{c++}
rf::Reference *Interpreter::method_call(ast::AST *root)
{
    std::string method_name = root->token.val_str;
    std::vector<rf::Reference*> computed_params;
    rf::Reference *return_reference;

    if(!root->children.empty())
        collect_params(root->children[0], &computed_params);

    call_stack.push_AR(method_name, lookup_method(method_name, root));
    ast::AST *method_root = call_stack.peek_for_root();
    init_record(root, &computed_params);

    if(method_root->children.size() == 2)
    {
        return_reference = exec_block(method_root->children[1]);
    }
    else
    {
        return_reference = exec_block(method_root->children[0]);
    }
    call_stack.pop();
    return return_reference;
}
    \end{minted}

    Method calls are a bit more complicated, as they require managing memory
    using a call stack, which holds activation records (AR) holding all variables
    from each function call. Every call requires an instance of AR
    pushed onto the stack. When the interpreter enters the 
    subtree of method call, the first thing it does is computing and 
    collecting arguments from function call to a vector of references. After that 
    an instance of activation record is allocated and pushed onto the stack. Function 
    \emph{lookup\_method} checks for declaration of called method and returns
    the pointer to the root of body of the method, which later is assigned to 
    variable \emph{method\_root}. Having done that, the record gets populated 
    with references by method \emph{init\_record} which assigns values to all 
    variables while also checking arity (number of supplied arguments). The 
    last if statement branches between functions with or without arguments
    because their structure varies. The \emph{block} node 
    (defined in grammar) is then passed to method executing blocks of statements 
    \emph{exec\_block} and assigning pointer to return reference to the 
    \emph{return\_reference} variable. After that, the top of the stack is 
    popped and the reference is returned. Recursion is possible owing to 
    \emph{exec\_block} method because it can call another method
    allowing for repeating the whole process with activation record pushed onto 
    the current one.

\end{document}
