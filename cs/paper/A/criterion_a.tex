\documentclass{article}
\usepackage[utf8]{inputenc}
\usepackage{hyperref}
\usepackage{url}
\usepackage[margin=0.75in]{geometry}

\title{Criterion A}
\author{Word count: 308}
\date{}

\begin{document}

\maketitle

\section{Problem}

    Pseudocode is an integral part of IB computer science curriculum. It is used
    to during lessons and during exams, so students need to be proficient at it.
    The problem is that the syntax of this pseudocode is quite strict, so during
    lessons we are assessing whether the code is correct manually. Despite there
    being an existing online interpreter for IB pseudocode to check for logic errors
    \footnote{link to existing interpreter: \url{http://ibcomp.fis.edu/pseudocode/pcode.html}}, it lacks in crucial
    feature which is error detection, which leaves students tracking subtle syntax
    errors for code to even run. Both myself and my friends have been frustrated 
    with this state of things, which forces us to waste time on mindless searching
    for errors that are purely syntactic and do not improve our skill [See 
    Appendix A for concrete complaints from my friends]. In order
    to remedy that situation, I came forward with proposition to build a 
    fully-fledged interpreter with lexical, syntax and run-time error detection.
    Due to there not being a strict standard for IB pseudocode, I would write
    the whole grammar for IB pseudocode, which could also serve as learning
    resource. Because I had to go outside the curriculum to find everything about
    how interpreters work, I constructed success criteria based on technical ascpects
    of the project.


\section{Success criteria}

\paragraph{Type of interpreter : tree-walk}
\paragraph{Language of implementation : \texttt{C++14}}
\paragraph{Features:}
\begin{itemize}
    \item executing IB pseudocode consistent with developed 
        standard based on documentation and exam booklets made by IB:
        \begin{itemize}
            \item methods
            \item recursion
            \item containers and ADT's
                \begin{itemize}
                    \item fixed-sized, multidimensional arrays
                    \item stacks 
                    \item queues
                \end{itemize}
            \item while and ``from to'' loops
            \item if else statements
            \item arithmetics with operator precedence
        \end{itemize}
    \item throwing lexical errors (undefined characters)
    \item throwing syntax errors
    \item throwing run-time errors
    \item flag to toggle printing abstract syntax tree
    \item flag to toggle logging and printing call stack
    \item cheat-sheet available with a flag
\end{itemize}
\paragraph{Additional goals:}
\begin{itemize}
    \item whole syntax defined in BNF (Backaus-Naur form)

\end{itemize}

\end{document}
