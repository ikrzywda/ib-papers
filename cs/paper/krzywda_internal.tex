\documentclass[12pt]{article}
\usepackage[utf8]{inputenc}
\usepackage{minted}
\usepackage{float}
\usepackage{hyperref}
\usepackage[margin=0.75in]{geometry}

\title{IBPCI - the IB Pseudocode Interpreter}
\author{Igor Krzywda}

\begin{document}

\maketitle

\section{Analysis}
\subsection{Problem}
The problem IBPCI solves is the lack of a dedicated tool for executing and debugging
IB pseuodocode. First objection one might have is that pseudocode by definition is
not meant to be executed and only used to present problems in a more approachable
way. This is true, but IB curriculum treats pseudocode as a proper programming language,
as it is required on the exams and even though it allows for more flexibility, still,
a student has to be familiar enough with it to write algorithms in a predefined syntax. 
\subsubsection*{Reality of learning pseudocode}
Practice in learning anything is key. The way pseudocode is meant to be learnt is to 
write it on paper, however, writing code on paper allows
for making mistakes that cannot be found very effectively making practice quite cumbersome,
as the student often can be ignorant of their mistakes when they are practicing on their 
own, and in class, it always takes simulating programs on paper, which is counterproductive
and a general waste of time. Considering that average IB student's schedule is quite
tight, it is not a good investment to practice writing code that cannot be easily tested
and will never be used outside school. An interpreter, which is IBPCI, would allow
students to quickly write and debug their code and refine their skills more efficiently.
\subsubsection*{Existing solutions}
Currently there are two paths one may take to execute their pseudocode. One is Dave 
Mulkey's web tool which converts pseudocode to JavaScript without and only analysis
done on the code is done during run-time in JS done by JS interpreter, all of which
is left out of sight of the user (it would not be of much help, as one would have to 
know how the code is translated to get meaningful information about errors in source 
code). Other route is to go on GitHub and search for pseudocode interpreter, however
none of the projects have meaningful documentation and are often not complete, so it
is not a feasable solution. 
\subsubsection*{Lack in standard}
A discrete goal for this project is to create an unofficial, concrete standard describing
grammars and behavior of the IB pseudocode. Such standard could be used for developing
different implementations of IB pseudocode. All of this is because the only information
about pseudocode can be found in either textbooks or data booklets. It would also free the information
about writing pseudocode from rather expensive official IB CS textbooks.
\subsubsection*{Target group}
Target group are students and teachers working with IB computer science curriculum for
the reasons mentioned above. 

\subsection{End goals}
\paragraph{Type of interpreter : tree-walk}
\paragraph{Language of implementation : C++14}
\paragraph{Functionality:}
\begin{itemize}
    \item executing IB pseudocode consistent with developed 
        standard based on limited documentation made by IB
        \begin{itemize}
            \item methods
            \item recursion
            \item containers and ADT's
                \begin{itemize}
                    \item fixed-sized, multidimensional arrays
                    \item stacks 
                    \item queues
                \end{itemize}
            \item while and ``from to'' loops
            \item if else statements
            \item arithmetics with operator precedence
        \end{itemize}
    \item throwing lexical errors (undefined characters)
    \item throwing syntax errors
    \item throwing run-time errors
    \item flag to toggle printing abstract syntax tree
    \item flag to toggle logging and printing call stack
    \item cheat-sheet available with a flag
\end{itemize}
\paragraph{Additional goals:}
\begin{itemize}
    \item whole syntax defined in BNF (Backaus-Naur form)
    \item documentation on behavior and functionality of the language
\end{itemize}

\section{Solution Overview}
\subsection*{Technologies}
\begin{itemize}
    \item \textbf{C++14} (g++ compiler)
        \begin{itemize}
            \item implementation language
            \item provides abstraction mechanisms without performance loss
        \end{itemize}
    \item \textbf{ASAN (ASan)} 
        \begin{itemize}
            \item debugging tool
            \item detects memory corruption (leaks, buffer and stack overflows)
        \end{itemize}
    \item \textbf{CMake}
        \begin{itemize}
            \item tool for generating makefiles 
            \item makes building larger C/C++ projects less cumbersome
        \end{itemize}
\end{itemize}
\subsection*{Test Plan}
In order to test success criteria, a series of programs will be written to test criteria
described in Analysis. Link to repository: \url{https://github.com/ikrzywda/ibpci/tree/master/examples}
\subsection*{Language design}
\subsubsection*{Tokens}
Tokens represented in regular expressions, which also doubles as a blueprint for lexer.
\begin{verbatim}
ID : [A-Z]([A-Z] | [0-9] | _)*
METHOD_ID : ([A-Z] | [a-z])[a-z]+ ([A-Z] | [a-z] | _)*
NUMBER : [0-9]+\.?[0-9]*
STRING : "."

RESERVED KEYWORDS / SYMBOLS:

method
if else then loop
while from to end
AND OR
+ - * / div %
< > >= <= == !=
( ) [ ]
\end{verbatim}
\subsubsection*{Grammar}
Backaus-Naur form (BNF) of IB pseudocode developed with reference to 
exam handout and pseudocode agenda. Since the parser is of recursive-descent type,
each terminal reflects how the parser builds abtract syntax tree.
\begin{verbatim}
program :: block
         | block method
         | empty

method_decl :: METHOD METHOD_ID (param_decl) block END METHOD

param_decl  :: ID
             | ID, param_decl
             | empty

block   :: stmt block 
         | stmt
         | empty

stmt    :: assignment
         | if
         | for loop
         | while loop
         | method_call
         | expr
         | return
         | std_method
         | BREAK

assignment  :: ID = expr
             | ID = condition
             | ID = arr_decl
             | ID = arr_empty
             | ID = std_method
             | ID = STACK
             | ID = QUEUE

if  :: IF condition THEN block END IF
     | IF condition THEN block ELSE if END IF
     | IF condition THEN block ELSE if ELSE block END IF
     | IF condition THEN block ELSE block

for_loop    :: LOOP id FROM expr TO expr block END LOOP

while_loop  :: LOOP WHILE condition block END LOOP

method_call :: method_id(params) 

return  :: RETURN expr
         | RETURN condition
         | RETURN

params  :: list_of_elements

condition   :: cmp AND cmp
             | cmp OR cmp
             | cmp

cmp :: expr > expr
     | expr < expr
     | expr >= expr
     | expr <= expr
     | expr == expr
     | expr != expr

expr    :: term + term
         | term - term
         | term

term    :: factor * factor
         | factor / factor
         | factor DIV factor
         | factor MOD factor
         | factor

factor  :: NUM
         | STRING
         | ID
         | method_call
         | std_method
         | arr_acc
         | (expr)

arr_decl    :: [list_of_elements]
             | [arr_decl, arr_decl]

list_of_elements    :: expr
                     | expr, list_of_elements
                     | empty

arr_empty   :: ARRAY(list_of_elements)

arr_acc :: ID accessor

accessor    :: [NUM]
             | [NUM] accessor
             | empty

std_method  :: OUTPUT(params)
             | INPUT(STRING)
             | container_method

container_method    :: ID.length()
                     | ID.push()
                     | ID.pop(params)
                     | ID.enqueue(params)
                     | ID.dequeue()
                     | ID.getNext()

\end{verbatim}
\section{Development}
\subsection*{Lexer}
Lexer  the part of a translator that converts text into tokens, which in case of 
IBPCI are instances of class Token. Token holds type a type and a value, which can
be either numerical or a string. Lexer simulates finite deterministic automaton that
based on current characted, goes into some state. In Figure~\ref{snip_1} is function
\emph{id}, which returns a token of variable id, method id or a keyword. The function
works by taking a character from the buffer and checking whether it is an alphanumerical
characted or an underscore (as stated in token definitions) and if it fits, it gets added
to a smaller buffer that holds token value. Because there is a distinction between
variable is and method id, there must be a variable that keeps track of what token lexer
is dealing with. By default this variable signifies variable name, but if there appears
a lower-case letter, its value starts to signify method name. Last test that must be
done is to check whether the lexeme is a keyword. In order to do that, method 
\emph{lookup\_keyword} gets called with buffer as its argument, which checks whether
the lexeme is a keyword in a map defined in \emph{Token} class. If it is a keyword, 
it returns the id of corresponding keyword, otherwise, it returns 0. At the bottom, 
the token variable inside \emph{Lexer} class gets mutated and the address of that
token is returned.
\begin{figure}[H]
    \caption{}
    \label{snip_1}
    \begin{minted}{c++}
        tk::Token &Lexer::id()
        {
            int id = tk::ID_VAR;
            attr_buffer.push_back(c);

            if(!is_upcase(c)) id = tk::ID_METHOD;

            advance();

            while(std::isalnum(c) || c == '_')
            {
                if(!is_upcase(c)) id = tk::ID_METHOD;

                attr_buffer.push_back(c);
                advance();
            }
            if(tk::lookup_keyword(attr_buffer) > 0)
                id = tk::lookup_keyword(attr_buffer);

            token.mutate(id, attr_buffer, line_num);
            return token;
        }   
    \end{minted}
\end{figure}
\subsection*{Parser}
Parser in the case of IBPCI is a recursive-descend parser where each terminal defined in 
grammar has its corresponting method. The goal of parser is to make an intermediate representation
(IR) of the source code. In the case of IBPCI the IR is an abstract syntax tree (AST). As an example
of how AST is built by the parser we will go through function \emph{expr()}
that returns, as the name suggests, an AST of an expression.
For better reference, we will use the grammar definition of expression in our language, 
as the parser function is a direct implementation of this rule.
\begin{verbatim}
expr    :: term + term
         | term - term
         | term
\end{verbatim}
Expression in Pseudocode is a an addition of terms, subtraction of terms or just a term.
This is caused by operator precedence, as addition and subtraction are at the end, and the 
interpreter evaluates and executes AST with depth-first tree traversal, so it makes sense for 
the addition and subtraction to be executed at the end. With that out of the way we can get to 
the code itself.
\begin{figure}[H]
    \caption{}
    \label{snip_1}
    \begin{minted}[linenos]{c++}
        ast::AST *Parser::expr()
        {
            ast::AST *root, *new_node;
            root = term();

            while(token.id == tk::PLUS 
                    || token.id == tk::MINUS)
            {
                new_node = new ast::AST(token, ast::BINOP);
                new_node->push_child(root);
                root = new_node;
                eat(token.id);
                new_node->push_child(term());
            }
            return root;
        }
    \end{minted}
\end{figure}
We start by declaring two variables that will hold poitners to AST nodes. We assign \emph{root}
pointer to \emph{term}, as it is the first nonterminal that has to be 
evaluated in each of cases in the grammar. Then, if the next token is neither \emph{PLUS}, nor
\emph{MINUS}, we return the node. In other case we enter the while loop that will turn as long as
the token during the check is \emph{PLUS} or \emph{MINUS}. Inside the loop, we assign a pointer to 
a subroot to variable \emph{new\_node}, which is the terminal of binary operation holding the operator.
The node currently held by \emph{root} is the left operand of binary operation and is linked to the root
by funtion \emph{push\_child} that appends the pointer to the vector inside the node holding the children.
After \emph{root} reclaiming the root of the tree, we advance to the next token using method \emph{eat}, which
checks whether the current token is the same as requested token by a method. Because we already checked if the 
token is correct, we just pass the id of current token. Lastly, we append another \emph{term} to our root and
either repeat the process or break out of the loop and return the poitner to the tree. This method, along with 
helper functions \emph{term} and \emph{factor} provided an example expression $10 + 100 * 10 - 6$ will output
following tree (actual output of the interpreter with flag \emph{-p}):
\begin{verbatim}
|--[-]               
    |--[+]        
        |--[10]   
        |--[*]     
        |--[100]
        |--[10] 
    |--[6]        
\end{verbatim}
\subsubsection*{Call Stack}
Becuse Pseudocode includes methods, so the memory of the program cannot be one big container, but rather, it
needs to be a stack. Call stack, as the name suggests is the stack that holds method calls. Because, as methods
are being called in program, each needs to get its activation record, which holds values of variables and upon calling
by interpreter, returns the values of variables requested by the interpreter. Call stack only manages activation records
and because methods are being executed sequentially, stack is the ADT of choice. \emph{CallStack} class is just a wrapper
for a literal stack, provided by standard C++ library, holding unique pointers to activation records. When an interpreter 
stumbles upon method call, an activation record is created and pushed onto the stack and executed, if there is another
method call from the within, another activation record is created and pushed onto the stack. All of this can go until 
stack buffer overflow, which means that the activation records have exceeded the RAM. 
\subsubsection*{Interpreter}
Interpreter is the part where the magic happens and the pseudocode comes to life. Interpreter of IBPCI is 
a tree-walk interpreter, which means that the code is directly evaluated and executed with reference to 
AST build by the parser. The interpreter does pre-order traversal on the AST and if the evaluation of the semantics is 
successful, the code gets executed. There will be two examples from the interpreter, one with method responsible for 
binary operations and other for method call.
\begin{figure}[H]
    \caption{}
    \label{snip_1}
    \begin{minted}[linenos]{c++}
rf::Reference *Interpreter::binop
        (rf::Reference *l, rf::Reference *r, int op)
{
    int type = check_types(l, r);
    rf::Reference *out;
        if(type == tk::STRING)
        {
            if(op == tk::PLUS)
            { 
                out = add(l, r);
            }
            else
            { 
                delete l;
                error("cannot make this type of comparison on strings", r);
            }
    }
    else if(op == tk::MINUS || op == tk::MULT || op == tk::PLUS)
    {
        switch(op)
        {
            case tk::PLUS: { out = add(l, r); break; }
            case tk::MINUS:
                 {
                    out = new rf::Reference(l->token.val_num - r->token.val_num); 
                    break;
                 }
            case tk::MULT: 
                 {
                    out = new rf::Reference(l->token.val_num * r->token.val_num); 
                    break;
                 }
        }
    }
    else
    {
        out = divide(l, r, op);
    }
    delete l; delete r;
    return out;
}
    \end{minted}
\end{figure}
In order to evaluate a binary operation, \emph{binop} takes as arguments left and right node of an expression subtree along
with the operator. This is because binary operation is bound to the right side of assignment, and because there are more
right-bound expressions, there is a function that manages that choice so that functions working with terminals can be 
cleaner and more readable. The first function, which is called is \emph{check\_types}, returns the type of expression, but
also has optional side effect of throwing a run-time error when incompatible types are to be computed. When the type check is 
successful, there are two main branches. If the type is a string, the interpreter is only to execute concatenation, which 
means that only viable operator is addition, which is tested in 6th line. If the types are numerical, then all binary operations 
can be performed. Addition, subtraction and multiplication are divided from divisions and modulo because they do not need checking 
for zero in right value. If any of the options checks out, a new reference (a field in activation record) is created with 
literal expression in its constructor. In the end, all heap-allocated, redundant objects are disposed of to prevent memory leaks and
pointer to reference is returned.
\begin{figure}[H]
    \caption{}
    \label{snip_1}
    \begin{minted}[linenos]{c++}
rf::Reference *Interpreter::method_call(ast::AST *root)
{
    std::string method_name = root->token.val_str;
    std::vector<rf::Reference*> computed_params;
    rf::Reference *return_reference;

    if(!root->children.empty())
        collect_params(root->children[0], &computed_params);

    call_stack.push_AR(method_name, lookup_method(method_name, root));
    ast::AST *method_root = call_stack.peek_for_root();
    init_record(root, &computed_params);

    if(method_root->children.size() == 2)
    {
        return_reference = exec_block(method_root->children[1]);
    }
    else
    {
        return_reference = exec_block(method_root->children[0]);
    }
    call_stack.pop();
    return return_reference;
}
    \end{minted}
\end{figure}
The most complex process when interpreting source code in undoubtedly calling methods. As explained with call stack, each function
call requires an instance of activation record pushed onto the stack. When the interpreter enters the subtree of method call, the 
first thing on the agenda is to compute and collect arguments from function call to a vector of references. After that an instance
of activation record is made and pushed onto the stack. Function \emph{lookup\_method} checks for declaration of called method and returns
the pointer to the root of body of the method, which later is assigned to variable \emph{method\_root}. Having done that, the record gets 
populated with references by method \emph{init\_record} which assigns values to all variables while also checking for arity (number of 
supplied arguments). The last if statement branches between function without arguments and one with arguments because their structure varies, as
the first node of AST of method with arguments are, well, arguments, and for one without, it is the body of the function which the 
intepreter is interested in. The \emph{block} node (defined in grammar) is then passed to method executing blocks of statements 
\emph{exec\_block} and assigning pointer to return reference to the \emph{return\_reference} variable. After that, the top of the stack is 
popped and the reference is returned. Recursion is possible owing to \emph{exec\_block} method because it can call another method
allowing for repeating the whole process on activation record pushed onto the current one.
\section{Appendices}
\end{document}
