\documentclass{article}
\usepackage[utf8]{inputenc}
\usepackage{float}
\usepackage{hyperref}
\usepackage{url}
\usepackage[margin=0.75in]{geometry}

\title{Appendices}
\date{}

\begin{document}
    
\maketitle

\section{References}
    
    \begin{itemize}
        \item Aho, Alfred, et al. \textit{Compilers: Principles, Techniques, and
            Tools}. Pearson, 1986
        \item Pivak, Ruslan. \textit{Let's Build A Simple Interpreter}, 
            \url{https://ruslanspivak.com/lsbasi-part1/}
    \end{itemize}

\section{Appendix A}

\subsection{List of complaints regarding using pseudocode I gathered from my friends}

    \begin{itemize}
        \item \textit{``I hate having to check the code myself, it's a waste
            of time.''}
        \item \textit{``The online tool we use is helpful, but the lack of 
            error detection annoys me to the point I ignore the tasks because
            I don't have the time to check every bit of code.''}
        \item \textit{``Pseudocode is easy, it takes a bit of practice and
            you can find bugs quite fast on your own, you can live without
            a dedicated tool.''}
    \end{itemize}


\subsection{Transcript of conversation with my friend Jan who was most interested 
        in my idea (translated to English)}

    \begin{itemize}
        \item Igor Krzywda (me): So I have this idea to build interpreter for 
            the pseudocode, what do you think?
        \item Jan Gumulak (JG): Don't we already have the tool for this, the one
            on the website?
        \item me: Yeah, I know, but I want my version to detect errors so that 
            we don't have to waste time looking for them manually. Would you use
            something like this?
        \item JG: Of course, I hate having to look for them, especially when during
            lesson everyone has some different error and we have to discuss every
            one before we move on. It takes ages!
        \item me: Ok, then, when I have something I'll send you link to my repo.
        \item JG: Sure, I'll check it out.
    \end{itemize}
   
\subsection{Transcript of the first meeting with my advisor - my computer science
        teacher}

    \begin{itemize}
        \item Igor Krzywda (me): Hello, I talked with others about my idea to make
            an interpreter for pseudocode which would detect errors in the code.
            Everyone seemed to be positive about the idea, so I did some deeper
            research and got general outline of what I'm going to do.
        \item Przemysław Pobiedziński (PP): Okay, that sounds good. What is the
            general outline?
        \item me: From what I read, the intepretation goes something like this:
            the source code in text needs to be divided into tokens, which are
            some identifier and a piece of data. Then all tokens need to be put
            into a tree so that in the end I can traverse it and do operations
            based on what is in any given node.
        \item PP: Seems correct, do you have resources to learn details from?
            Because you described the idea very generally.
        \item me: Yes, there are a lot of resources online, so I think I will
            be good on that front. Oh, I even managed to borrow the dragon book
            on compilers, it is supposed to be a really good textbook.
        \item PP: Sounds good. What do you want to write it in?
        \item me: I want to make it in plain C.
        \item PP: Are you sure? You are already taking on rather complex project,
            if you use C, you will have to write your own data structures and you
            will have to manage memory yourself. It will distract you from the main
            problem and will slow you down.
        \item me: I know, but I want to learn as much as possible, so I thought 
            that writing in more low-level language would be a good idea.
        \item PP: I understand you, but you have to keep in mind that you don't 
            have unlimited time so you take a big risk doing that. Why don't you
            use C++, it is a middle ground - you will have the access to programming
            on a low level of abstraction, but if you feel that it is getting too
            hard, you can use built-in containers. Not to mention that C++ handles
            strings, which I think in your case would be an immense benefit.
        \item me: Hmm... I think that you may be right. I will try to start writing
            in C, but if I feel that it is too hard for me, I will switch to C++.
        \item PP: Okay, do you have anything else you want to consult with me?
        \item me: No I don't think I have. Thank you for your time.
        \item PP: Sure.
    \end{itemize}

\section{Appendix B}

    Below is a list of tests done to test success criteria, source code is in 
    directory \texttt{product/ibpci/examples/tests}. The tests have also been done
    in the video.
    \begin{itemize}
        \item \texttt{data\_types\_demo.ib} - checking that all data structures,
            as well as I/O 
        \item \texttt{bubble\_sort.ib} - testing loops and conditionals
        \item \texttt{binary\_search.ib} - testing recursion
        \item \texttt{fizzbuzz.ib} - testing expressions in conditions
        \item \texttt{error\_demos/*.ib} - testing error detection, where filename
            describes expected error
    \end{itemize}


\section{Appendix C}

\subsection{Client feedback - general thoughts}

    \begin{itemize}
        \item \textit{``I like it. I intentionally put some errors and it
            detected them, even gave the line number!''}
        \item \textit{``Everything worked great, but I wish you made a binary,
            it took me a lot of time to set everything up, though I feel that
            it was a good investment considering that we will be doing a lot
            of pseudocode''}
        \item \textit{``It took a bit of figuring out with using command prompt,
            but I used it and it was really nice to see errors pop up. Think
            about making a GUI version, because not a lot of people will use
            it on Windows otherwise.''}
        \item \textit{``Man, I tried but didn't have the time to set everything
            up to compile the project. Make a Windows binary and I'll surely try
            it''}
        \item \textit{``It worked quite well, but when I tried to compute a factorial
            recursively it crashed completely, something doesn't work with that''}
    \end{itemize}

\subsection{Client feedback - conversation with Jan}

    \begin{itemize}
        \item Igor Krzywda (me): So what do you think?
        \item Jan Gumulak (JG): I like it, I haven't yet written code that crashed
            it, so I guess that's not bad.
        \item me: So I guess you didn't tinker a lot with recursion yet. What
            about errors, are they useful?
        \item JG: Oh, they are great, especially syntax ones. Run-time errors
            sometimes give strange descriptions, one time I got comparison
            between null and string.
        \item me: Hmm... I'll look into that, thank you. Have you tried the flags?
        \item JG: Yes, but I don't think that they are a nice quirk. The one 
            displaying tokens is interesting, same with the tree, but if someone
            was to learn from that, you need to make some documentation. The stack
            flag, on the other hand, was quite useful, I was tracking the variables
            in the functions. You could make it a bit cleaner, though, it is not
            very readable.
        \item me: I'll work on that, thanks.
        \item JG: No problem.
    \end{itemize}
   
\subsection{Final meeting with advisor - transcript}

    \begin{itemize}
        \item Igor Krzywda (me): Hello, I sent you the source of the project and
            success criteria. Could you try it and compare it with success criteria?
        \item Przemysław Pobiedziński (PP): Sure, I will try it and get back to 
            you.
        \item me: Okay, thank you.
        \item PP: I tried the test programs you included and everything works 
            like in the film, so the most of the success criteria you set check
            out. However, there is no cheat-sheet you gave in the list. Also, I 
            tried some recursive algorithms, and some of them worked, but when
            I tried returning an expression containing function call, I got an 
            error, so you need to work on that. Over all I think that everything
            in terms of goal you set yourself is fine.
        \item me: Thank you, do you have any extra suggestions or problems you 
            came across outside the success criteria?
        \item PP: Yes, I think that you shuld make a graphical interface for others
            to be able to use more easily. The program is good, but it is not
            very user-friendly, if you made a GUI it would be a great tool to use
            during lessons, but at this moment it is too hard for someone with 
            no experence to use, not to mention compiling it. Another thing I 
            would add are generic collections, notice that in the exercises you 
            need to use a collection, not a stack or queue. 
        \item me: Thank you for your feedback, I will keep it in mind when I will
            be playing with the program.
    \end{itemize}


\end{document}
