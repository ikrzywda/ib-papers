\documentclass{article}
\title{The effect of color on performance in logic-based tasks}
\author{Igor Krzywda}
\begin{document}
\maketitle

\section*{Introduction}

\subsection*{The theory behind the research}
	The main theory coming into this field of research is Kurt Goldstein's (1942) theory claiming that
	the wavelength of light affects human organism physiologically as well as psychologically. This theory
	says that colors with the longer wavelengths (red / yellow / orange) have stimulating effect on people, 
	i.e. it gets us in unsettled, more energized state, whereas colors with shorter wavelengths (blue / green)
	have calming and relaxing effect on us. According to this theory, our hypothesis is that the tests done
	with red / yellow color manipulation shoud yield worse results than those with the blue / green ones.

\subsection*{The original experiment}
	The replicated experiment is one of the experiments used in research on 'Color and Psychological Functioning
	The Effect of Red on Performance Attainment' by Andrew J. Elliot, Markus A. Maier, Arlen C. Moller, 
	Ron Friedman and Jörg Meinhardt (2007). The experiment involved a group of 71 undergraduate students
	asked to fill anagrames, f.e. elbow → below. The practice was done on plain paper with no accents. After 
	the practice, the participants were handed out the same questionaires, but this time each had a colored
	label with a number on it. Participants were asked to verify their numbers and then to do the test. The 
	results showed the correlation between the color and the performance: red group scored the lowest at 4.5
	mean correctly solved anagrams, green and black scored the same at 5.5. 

\subsection*{Investigation statement}
	The aim of the study is to find out how does the color affect the performance in logic-based tasks. 
	The applications of the results of this study are very broad spanning from student's performance at
	school to a programmer's effectiveness at writing code. This study can give a better understanding how 
	to better manipulate our physical environment (f.e. color of the paint on the walls) as well as digital one
	(software themes) to suit our needs from focus to elevated mood.

\subsection*{Hypotheses}
	\begin{itemize}
	\item{Null hypothesis}
	\begin{itemize}
	\item{The color of the labels has no effect on the mean number of anagrams correctly solved}
	\end{itemize}
	\item{Research hypothesis}
	\begin{itemize}
	\item{The color of the labels has effect on the mean number of anagrams correctly solved}
	\end{itemize}
	\end{itemize}

\section*{Exploration}

\end{document}
